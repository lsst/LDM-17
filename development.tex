% section 3: Software Development Practices

\section{Software Development Practices}

The development practices of the LSST development team were 
described in detail in the report on Data Challenge 2. This
section restates some central points and describes some new
features of the software development environment.


\subsection{Technical Control Team}

A goal of DC3 was to increase the formalization of the
development process as a way to compensate for the increasing
size of the increasingly distributed development team.
Accomplishing this goal called for a more central role for
the Technical Control Team (TCT), known in DC2 as the
Configuration Control Board (CCB), which serves as an internal
forum on key topics for the development of LSST Data Management software, including 
major design issues, the tools to be used, and standards and policies.
The TCT meets monthly to set development policy 




(settling 
questions such as, when are developers expected to work
on branches or 



, come to agreement on 

approval of the use of third-party software packages,



\subsection{UML Modeling}

UML modeling continues to play a central role in the code
development process. 

Software classes for LSST are designed in UML, the Unified
Modeling Language, before coding. UML is a general set of 
standards for diagramming abstract models for object-oriented
development. There is a general LSST model meant for the
final production code, and specializations of that model for
the specific subset of that model that falls within the scope
of the current data challenge. At the end of each data
challenge, the general LSST model is updated with 

As in previous data challenges, the UML models are created
and maintained using a commercial software package, 
Enterprise Architect, allowing for interactive and collaborative
design.

New components or software stages are first designed in UML,
and before any code is written the model is reviewed 

\subsection{Software Development Environment}





This included the creation of an automated ``buildbot'' which
on a daily basis builds the hardware stack from scratch.

\subsection{Coding and Documentation Standards}

Code standards for C++ and Python were 


