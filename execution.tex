% section 5.1.2: Pipeline Execution

\subsubsection{Pipeline Execution}


Exception throwing and handling was improved in DC3a. The interface was
simplified by removing the virtually-unused inheritance from {\tt
DataProperty}.  Instead, five features were added:

\begin{itemize}

\item A simple string message is used as the primary payload of the
exception, for compatibility with standard C++ and Python exceptions.
This also makes throwing an exception simpler for the programmer.

\item A combination of macros and classes was used to automatically
include the file, line, and function where the exception was thrown.
This feature improves the debuggability of the code.

\item Other macros allow arbitrary additional parameters to be added to
subclasses of the generic {\tt lsst::pex::exceptions::Exception} class.

\item Further macros were defined that allow additional traceback
information, including additional messages, to be added to a caught and
rethrown exception.

\item LSST C++ exceptions are transformed via SWIG code into instances
of the {\tt LsstCppException} class in Python, which inherits from the
standard Python {\tt Exception} class.  The underlying SWIG-wrapped C++
exception is available as an argument of the {\tt LsstCppException}, and
the C++ exception's message is automatically included as part of the
Python exception's message.  Previously, LSST C++ exceptions were
transformed into Python exception classes that did not inherit from the
standard Python {\tt Exception} class.

\end{itemize}

The result was an exception design that was simpler to throw and that
targeted the location of the exception much more precisely.  Programmers
are still not using the more advanced features of the exception facility
such as re-throwing or additional parameters; if they are found to be
unneeded, the class can be further simplified in the future by removing
them.

