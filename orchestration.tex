% section 5.1.3: Pipeline Orchestration

\subsubsection{Pipeline Orchestration} \label{sec:PipelineOrchestration}

The orchestration layer is a collection of Python objects used to set up and
launch pipelines. This collection of objects is invoked via a command line
utility named \texttt{Orca}.

\texttt{Orca} takes as arguments a run ID and a policy file. The run ID is used
to identify the pipeline run. The policy file lists configuration information
about the platforms where pipelines are to be launched, the type of database to
use, and the event broker to use.

\texttt{Orca}'s object hierarchy is architected so that platform specific work
is handled by subclasses and more generalized work is handled by super-classes.
The objects that \texttt{Orca} instantiates and uses are specified by policy
files. As we add support for new platforms and database configurations, we can
expand this object hierarchy, customizing platform specific tasks to the methods
that will be executed without having to rework the main code for \texttt{Orca}.

The three main objects that \texttt{Orca} uses are the
\texttt{ProductionRunManager}, the \texttt{DatabaseConfig\-urator} and the
\texttt{PipelineManager}.  The \texttt{DatabaseConfigurator} and the
\texttt{PipelineManager} objects are subclassed to provide database and platform
specific functions.

The \texttt{ProductionRunManager} looks up in the policy the type of
\texttt{DatabaseConfigurator} to use and instantiates the appropriate object.
The \texttt{DatabaseConfigurator} is responsible for looking up the appropriate
database authentication information for the user, connecting to the database,
and creating the necessary database tables that will be used to record
information about the run.

The \texttt{ProductionRunManager} then creates a \texttt{Provenance} object used
to record information to the database about the provenance about the run. This
provenance information includes all information in the command line policy file.

The \texttt{ProductionRunManager} uses the policy to lookup and create platform
specific \texttt{Pipeline\-Manager} objects to configure and run pipelines. The
pipeline policy and software environment provenance is recorded for each
pipeline to the database.

The \texttt{PipelineManager} that is invoked is specified in the platform
section of the policy for that pipeline.  This section describes the root of the
pipeline directories, the names of directories to create, and the pattern to use
to create those directories.  It also describes the hardware configuration to
use, and the nodes available for use by a pipeline. The \texttt{PipelineManager}
places all the files required for the runs in locations where the pipelines can
get to them.

Once all the policy and data files are put into place, the Production run
manager has each \texttt{PipelineManager} launch its pipeline.
