\subsection{Application Framework}

As in DC2, the application framework is a library of
scientifically-oriented C++ classes that support the development of
astronomical processing algorithms.  A general summary of the library
can be found in the DC2 report (section 4.3).  As part of the code
reorganization at the start of DC3a, the package containing this
library was changed to {\tt afw} (with the corresponding namespace 
changed to {\tt lsst::afw}), but its contents generally remained the
same, apart from some new classes added and an over-haul of the image
API.  The new classes support the newest algorithms and are discussed
later in section \ref{sec:applayer} (e.g. \code{Statistics},
\Sec{secBackground}; \code{Psf}, \Sec{secPsf}).  In this section, we
focus on the changes to the image API.  

We note that we consider it an important goal to establish a stable
and robust {\tt afw} package during the current research and
development phase as this package provides the core building blocks of
all our higher level code.

\subsubsection{The Image Classes}
\label{secImageClasses}

In DC2 pixels in images and masks were managed using a toolkit from NASA-Ames, the ``Vision Workbench'' (\code{vw}).
Access to the pixels in an image was achieved by explicitly following a pointer to a \code{vw} class,
thereby exposing all of \code{vw} API, including any incompatible changes that might appear in
the future.

A further major problem with \code{vw} is that its concept of pixel access is restricted to a simple
STL-like iterator, whereas many astronomical algorithms require examination of the neighbourhood of
a pixel.

At the end of DC2 we therefore decided to rewrite all of our image classes to use \code{boost::gil}
(an image library originating at Adobe) which has a richer set of ways of accessing pixels;  in
order to preserve encapsulation, we added a thin wrapper layer above the raw \code{boost::gil} calls.
Another major goal of the rewrite was to make the code that manipulated the pixels of a single
\code{Image} identical to that that manipulates a \code{MaskedImage}'s pixels --- thereby enabling
generic programming of e.g. convolutions.

All of these goals have been achieved for DC3a.  The new APIs have proved to be convenient and
powerful, and the code generated is efficient,\footnote{at least with g++ versions $>= 4.2$} even when
manipulating images, masks, and variances simultaneously when processing \code{MaskedImage}s.

