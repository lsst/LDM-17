\documentclass[preprint]{aastex}
\usepackage{graphicx}

\newcommand{\code}[1]{\texttt{#1}}
\newcommand{\Fig}[1]{Fig.~\ref{#1}}
\newcommand{\Sec}[1]{Sec.~\ref{#1}}
\newcommand{\Table}[1]{Table~\ref{#1}}
\newcommand{\ticket}[1]{ticket \code{\##1}}
\newcommand{\by}{$\times$}

\shorttitle{LSST Data Challenge 3a}
\shortauthors{Axelrod, et al.}

\DeclareGraphicsExtensions{.pdf,.jpg,.png,.eps}

\newcommand{\XXX}[1]{{\bf XXX #1}}      % Please use this for anything that needs to be written;
                                        % we can comment the definition and find all the \XXXs

\newcommand{\RHL}[1]{{\bf RHL: #1\qquad}}     % A comment from RHL.  None should remain in the final document!

% -- a @%< hack to redefine placement of abstract in \maketitle
\makeatletter
\def\@maketitle{%
 \newpage
 \begingroup
  \let\footnote\thanks
  \let\email\make@authoremail
  \let\affil\make@affil
  \let\altaffilmark\make@altaffilmark
  \let\altaffiltext\make@altaffiltext
  \let\and\make@and
  \@title
  \@author
  \@date
  \par
  %\@abstract
  \@ifxundefined\keyword@list{}{%
   \expandafter\@keywords
   \expandafter{\keyword@list}%
  }%
 \endgroup
 \clearpage
}%
\makeatother


\begin{document}

\title{LSST Data Challenge 3a}


% -- Author/Affiliation information
\author{Tim Axelrod, Robyn Allsman, Jeff Kantor, 
        Francesco Pierfederici}
\affil{LSST Corporation}

\and\author{Serge Monkewicz, Russ Laher, Deborah Levine, Vince
  Mannings, Jeonghee Rho, Peregrine McGehee}
\affil{NASA Infrared Processing and Analysis Center}

\and\author{Greg Daues, David Gehrig, Steve Pietrowicz, Raymond Plante}
\affil{National Center for Supercomputing Applications}

\and\author{Jacek Becla, Kian-Tat Lim}
\affil{SLAC National Accelerator Laboratory}

\and\author{Andrew Becker, Andrew Connolly, Russell Owen, Nicole Silvestri}
\affil{University of Washington}

\and\author{Steven Bickerton, Robert H. Lupton, Fergal Mullally}
\affil{Princeton University}

\and\author{Richard A. Shaw}
\affil{National Optical Astronomy Observatory}

\begin{centering}
\textbf{ABSTRACT}

\begin{quotation}\noindent%
This report describes the third in a series of formal prototypes of
the LSST Data Management System referred to as Data Challenge 3a (DC3a).
The focus of this data challenge is to enhance both the science applications
and the software infrastructure supporting those applications to add
increased functionality as guided by our analysis of the results of DC2. 
We enumerate the goals and metrics for DC3a, summarize the architecture
of the DC3a pipeline system, and review the science application and middleware
components that make up the system.  Finally, we present the results
of executing DC3a on data from the CFHT Legacy Survey and simulated
exposures created for LSST Data Management, and highlight the important 
conclusions that will be used as input into the next data challenge.  
\end{quotation}
\end{centering}

\tableofcontents

% -- section 0
\section*{Executive Summary}
\addcontentsline{toc}{section}{Executive Summary}

\subsubsection*{Results}

\subsubsection*{Middleware and Infrastructure}

\subsubsection*{Development Environment}

\subsubsection*{Required software developments for DC3b}

\subsubsection*{Scaling to LSST}




\pagebreak

% -- Section 1
% section 1: Introduction

\section{Introduction}

\subsection{Goals of DC3a}

\subsection{Metrics and Validation of DC3a}




% -- Section 2
% section 2: Components

\section{Components}

Harness

Pipelines

Database

Events Broker

\subsection{Overview of Processing Flow}

As in DC2, processing flow is guided by three production pipelines. Each of
these pipelines will be explained in greater detail below.

\being{itemize}

\item The \textit{image processing and source detection} (IPSD) pipeline takes 
as input the raw exposure images along with the related calibration data and 
produces a catalog of the light sources found within those images.

\item The \textit{Moving Object Processing System} pipeline (MOPS) takes
time and coordinate information from the exposure and creates
a catalog of known solar system objects expected to be within
the exposure FOV.

\item The \textit{association pipeline} (AP) then correlates the sources found
by the IPSD with known objects --- either fixed objects or objects expected
by MOPS to be within the FOV --- to determine whether unexpected sources
have been detected.

\end{itemize}

\subsubsection{Image Processing and Source Detection (IPSD)}

The IPSD pipeline processes images in pairs: two exposures, designated 0 and 1, 
are considered to have been taken consecutively of the same field of view at 
essentially the same time. This pairing of exposures allows for the detection 
of cosmic rays appearing in one exposure but not the other.

In order, the IPSD does the following:

\begin{itemize}

\item Information is read in identifying the exposures to be processed, and 
links are created in the file system from the working directory to the input
images.

\item For image 0 of the pair, metadata in the FITS file header is read in, 
giving the exposure time, location of the field of view, and similar information, 
which is then persisted to the clipboard.

\item Given this information, the calibration products associated with that
exposure are identified by lookup. Calibration products include darks,
flats, bias, and scatter exposures associated with a given camera run.

\item These calibration products are then used to perform instrument
signature removal (ISR) on the raw exposure, resulting in a calibrated image.

\item Source detection is then performed on calibrated image, giving source
information needed later for determining the WCS coordinates of the
exposure.

\item The point spread function (PSF) of the exposure is determined.

\item The second image of the pair is read in and also goes through ISR.

\item WCS of the exposures is determined.

\item The calibrated exposures are then persisted.

\item A template image representing the ideal expected exposure is
created. When processing CHFT-LS images, these templates are 
resampled from stacked images provided by the CHFT-LS survey.

\item Calibrated images 0 and 1 are subtracted from the template
image, and the difference images created are persisted.

\item These difference images are added together. Source detection
and measurement is then performed. The sources detected are
persisted in the database.

\item The association pipeline is signaled through the events broker
that the processing of the image pair is complete and that the source
data is ready for the association process.

\item SDQA data from the exposure is persisted.

\end{itemize}

\subsubsection{Moving Object Processing System (MOPS)}

\subsubsection{Association Pipeline (AP)}

\subsection{Applications}

\subsection{Hardware Deployment}

\subsubsection{The NCSA LSST Cluster}

Most of the DC3a preliminary and production runs were performed on a dedicated 
ten-nodes Dell Server Xeon cluster hosted at NCSA. This heterogenous cluster 
consists of two Xeon 3.6 GHz single dual-core nodes, two Xeon 3.6 Ghz dual
dual-core nodes, and six Xeon 2.0 Ghz dual quad-core nodes. It uses a
gigabit ethernet interface.

This cluster was built for DC2 and the DC2 runs were performed there; for DC3, 
the cluster was upgraded from 32-bit Red Hat Enterprise Linux 4 to 64-bit
Red Hat Enterprise Linux 5. Each cluster node has 4 GB memory, with the exception
of \texttt{lsst10}, which has 16 GB. Each node has from 20 to 60 GB of local disk,
along with 15 TB of disk storage shared among nodes using the NFS shared file system.
As part of DC3a, we installed the Lustre parallel file system to increase I/O 
throughput. We were not, however, able to stabilize the Lustre installation
sufficiently to use it for production runs on the NCSA cluster, as planned.

\subsubsection{NCSA Abe}

For purposes of testing the scalability of the pipelines, additional runs were
dual quad-core Xeon 2.33 GHz processors; for the LSST runs we used 36 of these 
nodes, for a total of 288 cores, sufficient to process an entire focal plane 
for the CHFT-LS images. Each core has 1 GB of memory, and shared access
to 100 TB of disk storage as managed using Lustre.

Abe job runs were coordinated using the Condor-G jobs management system.
The events broker and database were not moved to Abe during these runs;
they remained on the LSST cluster.

% -- Section 3
% section 3: Software Development Practices

\section{Software Development Practices}

The development practices of the LSST development team were described in detail
in the report on Data Challenge 2. This section restates some central points and
describes some new features of the software development environment.

\subsection{Technical Control Team}

LSST DMS is a distributed development project stretching from Princeton to UC
Davis. This geographical distribution makes it important to have
well-considered, well-documented coding practices.

A goal of DC3 was to increase the formalization of the development process as a
way to compensate for the increasing size of the increasingly distributed
development team. Accomplishing this goal called for a more central role for the
Technical Control Team (TCT), known in DC2 as the Configuration Control Board
(CCB), which serves as an internal forum on key topics for the development of
LSST Data Management software, including major design issues, the tools to be
used, and standards and policies. The TCT meets monthly to set development
policy on matters such as consideration and approval of LSST coding standards,
overall LSST package boundaries (that is, at the broadest level, determining
which component in the LSST stack is responsible for which functionality), and
the adoption of third-party open source software packages for use within the
LSST stack.

\subsection{Open Source Software: Use and Distribution}

The use of third-party open source packages is essential for LSST; these
packages represent standard off-the-shelf solutions for many of those LSST DMS
requirements which are not unique to LSST, freeing developers to concentrate on
those areas of LSST DMS for which no off-the-shelf solution exists. But these
packages must be carefully considered, and they must be agreed upon by the TCT.
In some rare cases, the TCT can withdraw its acceptance of a package. After DC2,
for example, it was judged that the CORAL and SEAL third-party packages
(developed by CERN) should be removed from the LSST stack, primarily because the
difficulty of building these packages from source outside the CERN-approved
Scientific Linux environment in which it was developed forced us to distribute
it as a pre-compiled binary, severely limiting the platforms on which the LSST
DC2 stack could run. The TCT determined that the functionality CORAL and SEAL
provided could be provided more effectively and flexibly with a comparatively
small amount of LSST-created code, thereby releasing the platform constraints.

The LSST stack is itself licensed, by TCT determination, under the GPL3 Gnu
General Public License\footnote{http://dev.lsstcorp.org/trac/wiki/SWLicense}.
GPL3 allows others to copy, redistribute, extend, or modify the software as long
as source code for such extensions is still distributed in source form. It does
not allow the software to be used as part of a proprietary software package.


\subsection{UML Modeling}

UML modeling continues to play a central role in the code development process.
UML is a set of conventions for diagramming abstract models
for object-oriented development. There is a general LSST model meant for the
final production code, and specializations of that model for the specific subset
of that model that falls within the scope of the current data challenge. 

UML design of a component is driven by use cases, the scenarios under which the 
component will be used. As in previous data challenges, the UML models are 
created and maintained using a commercial software package, Enterprise 
Architect\footnote{http://www.sparxsystems.com.au/}, allowing for interactive
and collaborative design. The Enterprise Architect installation is hosted
by LSST in Tucson, but can be accessed remotely using remote desktop viewing
via VNC (Virtual Network Computing) utilities. 

New components or software stages are first designed in UML, and before 
any code is written the model is given a design review. 
At the end of each data challenge, the general LSST model is updated using 
the UML documents created as part of that data challenge; the designer
indicates which aspects of the UML design for the data challenge are intended
to become a permanent part of the general LSST model and which aspects are
for that data challenge alone.


\subsection{Software Development Environment}

One new form of software quality assurance brought into DC3a was the use
of automated testing of code builds. This involved the use of an 
automated ``buildbot'' which on a daily basis builds the software stack 
from scratch, and builds all components from the LSST software trunk against
a standard stack of utilites, producing a report via web 
interface\footnote{http://dev.lsstcorp.org/buildbot/} for displaying the
results of a build. If someone checks in a package which doesn't 
compile correctly against the LSST stack, the buildbot will send out
a warning email. A waterfall display shows a graphical timeline of builds; if a
package build fails, that package is displayed in red on the timeline.
The LSST buildbot was implemented using the standard
open source package \texttt{buildbot}\footnote{http://buildbot.net/trac}
and some LSST-specific scripts.

For DC3b, it is anticipated that the buildbot will be expanded to include
automated checking of the package source code to verify compliance with
the associated LSST coding standard. The TCT has investigated several
static analysis tools for C++ standards checking. Such tools can verify
compliance with many of the LSST coding standards, although some of
our standards (such as 3-4 of the C++ standard, which calls for function
names to be verbs) cannot easily be checked by machine and, therefore,
total automated coverage of our coding standards is unlikely.

\subsection{Software Integration Schedule}

There is a strong sense among the development group that the software
integration process for DC3a had a significant flaw. For previous data
challenges, DC1 and DC2, teams independently developed code on their own
platforms and then brought the components together for during an ``integration
week,'' when a large segment of the development team gathered at NCSA for a week
and worked through the integration process in a tiger-team fashion. For DC3,
this had an unfortunate consequence, in that there was effectively no test bed
during the initial stages of coding, since until the integration phase
there was no system running all of the components necessary to complete 
even a partial pipeline run. (Substantial code refactoring and the move
from a 32-bit to a 64-bit operating system for development meant that the 
DC2 stack could not simply be frozen and used for this purpose.)

For the purposes of DC3a, a short-term solution was provided by
\texttt{simpleStageTest.py}, a simple harness allowing some application stages
to be executed independently. This allowed for some additional debugging
capabilities.

Following DC3a, it was felt to be essential that an installation capable
of running the pipelines be maintained throughout the coding process,
so that pipeline testing --- particularly, the assessment of the 
science value of stage results, and the ability to interactively tune
the parameters of stage algorithms --- can be performed in an on-going fashion
rather than being deferred to the last stages of the data challenge.



% -- Section 4
% section 4: Input Data

\section{Input Data}


\subsection{CFHT-LS Deep Survey Data}

32 CCDs.

\subsubsection{Science Data Products}

The ``raw'' archival science data obtained from CFHT have  gone through
a modicum of pre--processing by the CFHT {\tt ELIXIR}
pipeline\footnote{\url{http://cfht.hawaii.edu/Instruments/Imaging/MegaPrime/rawdata.html}}.
In particular, the two amplifier readouts were spliced back together
into 2112 by 4644 pixel images corresponding to a single CCD.  To prepare the data for DC3a processing, we need
to undo this operation to some degree, while still maintaining
consistency between the valid data sections, overscan regions, and
approximate Wcs information in the image headers.

For DC3a processing, we divided each of the 32 CCD images into 8 images
that are approximately the anticipated size of LSST amplifier images.
Synthesized amplifier images 0-3 come from CFHT image subsection {\tt
[0:1056,0:4612]} of the input images, and amplifier images 4-7 from
{\tt [1056:2112,0:4612]} (there is an additional overscan region from
$4612 \leq y \leq 4644$ that is ignored).  Each resulting amplifier
image is 1056 x 1153 pixels, with 32 pixels of overscan implemented as
``prescan'' (amplifiers 0-3) or ``postscan'' (amplifiers 4-7).  When segmenting the images, we adjust the following Metadata keywords :

% NOTE - I coped these sections straight out of 
% prepdc3a/python/stageCfhtForDc3a.py.  Unfortunately
% these data sections end up being a mix of python slicing 
% (0-indexed) and IRAF/FITS convention (1-indexed).  How did I 
% deal with the fact that the sections are originally defined 
% in this convention?  This is dealt with in Isr.cc, BBoxFromDatasec.

\RHL{Do we need this much detail?}

\begin{itemize}

\item {\tt RDNOISE} : Each amplifier has a different amount of
readnoise.  In the spliced CFHT CCD image, these are recorded as {\tt
RDNOISEA} and {\tt RDNOISEB}.  For our segmented amplifier images we
assign {\tt RDNOISE} as either {\tt RDNOISEA} (amplifiers 0-3) or {\tt
RDNOISEB} (amplifiers 4-7).

\item {\tt GAIN} : Derived from {\tt GAINA} and {\tt GAINB} in a
manner similar to the {\tt RDNOISE} field.

\item {\tt BIASSEC} : This is set to either columns {\tt [1:32,]} (amps
0-3) or {\tt [1025:1056,]} (amps 4-7).

\item {\tt DATASEC} and {\tt TRIMSEC} : This is set to either columns
{\tt [33:1056,]} (amps 0-3) or {\tt [1:1024,]} (amps 4-7).

\item {\tt CRPIX1} : This is left as--is for amps 0-3, and corrected
by -1023 pixels for amps 4--7.  Additionally, for CCDs 1--16, amps
0--3 are adjusted by another -33 pixels, and for CCDs 17--32 amps 4--7
are adjusted by another -33 pixels.  This reflects the exclusion of a
secondary overscan region at the ``top'' or ``bottom'' of the CCD,
depending on the orientation of the CCD on the focal plane.

\item {\tt CRPIX2} : This is adjusted by the y--axis offset of each
amplifier image from the original CCD image.

\end{itemize}

The visitId associated with each Image comes from the Metadata keyword
{\tt OBSID}, and we assume that this image represents the first
exposure (exposureId = 0) of the cosmic ray split.  The second exposure of the visit (exposureId = 1) 
is synthesized by taking the actual
CFHT image and adding a set of cosmic rays.  The segmented
files are saved on disk as Images using the following formatting :
{\tt 'raw-\%06d-e\%03d-c\%03d-a\%03d.fits' \% (visitId, exposureId,
ccdId, ampId)}

\subsubsection{Calibration Data Products}

The CFHT calibration data includes FITS keywords that define the range
of dates for which it is valid.  In order to process CFHT science frames
we needed to associate a given filter/date with the proper calibrations.  We
accordingly generated a \texttt{.paf} (i.e. \texttt{Policy}) file that may
be read into a \texttt{Policy} with nested fields that enable the fast
lookup of the correct calibration file.

According to Astier, ``the Elixir flats that you download from CADC
are not fully correct on large scales. Namely, the flux of the same
star still varies by a few percent center-to-corner'';  we have not
attempted to correct for this effect in DC3a,  although it is clear
that the final LSST system will need to do so.

\subsection{LSST Simulated Images and Catalogs}

AJC.

\subsubsection{Science Data Products}
AJC and JP.
% JP
The simulated images are generated from an input catalog.  The 
input catalog includes the position in the sky (right ascension \& declination), 
apparent magnitudes at 5500 \AA, a Spectral Energy Distribution file, a
redshift, and when the object is a galaxy it includes the morphological
information.  The catalog is input into the raytracing software which 
outputs a FITS image, which is a single CCD.  The raytracing software requires 
input that includes the atmosphere conditions, optics perturbations, observed
airmass, and LSST filter.  After the raytracing software has generated a 
single CCD Fits image, background and the associated noise is added to 
each image.  The background calculation includes the night-time sky emission 
along with the lunar emission.  The final processing steps include the 
calculation of the TAN-SIP WCS polynomials, and updating the FITS header 
which includes WCS information and book keeping information.  Amplifier
information is made  by breaking up each CCD into 16 amplifier 
images (512 $\times$ 2045 pixels), an overscan region of 5 pixels is added 
along the long column, ansd read noise of 5 elections is added to all pixels.
A software bias of 100 counts is added to all the pixels to avoid negative 
counts.  All amplifier-level FITS images are in 16-bit unsigned integer 
format.  The key words include the simulation run name (i.e. Wide, Deep, 
or Ideal), dates processed at University of Washington and generated at Purdue, 
along with CCD processing information (i.e. DATASEC, etc..) 

information about Deep, Ideal, Wide? Purdue run time, CONDOR?
Are these products or runs?


\subsubsection{Calibration Data Products}

AJC and JP.
%JP
Calibration data products are flat field images, dark frames, and bias 
frames. The flat field images are currently set to unity. The dark 
and bias frames are currently zero.  



\subsection{Event Generation From Input Data}

The input data exist on disk as Fits files, with all requisite
metadata in their headers.  Conversely, the paradigm for LSST is to
have the pixel--level science data and associated Metadata as separate
entities, the latter ideally existing in a database but in any case
being received separately from the camera system.  Therefore to
generate an event for DC3a processing, we must first address the
images on disk, translate their Fits metadata fields (e.g. 'GAIN')
into LSST--format metadata fields (e.g. 'gain'), assemble selected Metadata
representative of what the camera would know into a PropertySet, and
send this as Event to the pipeline.

This process required a metadata $\rightarrow$ Metadata mapping for each dataset.
This was implemented as so.


% -- Section 5
\section{Software Framework}

Much of the LSST software stack is provides called \textit{framework}
classes: reusuable building blocks from of which our pipelines are
built.  In this section, we will review the major framework classes
via three main categories:  middleware, database, and
application.  Much of the introductory information describing the
framework packages can be found in the DC2 report.  In this report, we
will focus on the changes since the DC2 implementation.  

\subsection{Middleware Framework}

% section 5.1.1: Data Access

\subsubsection{Data Access}





% section 5.1.2: Pipeline Execution

\subsubsection{Pipeline Execution}


\subsubsubsection{Exceptions}

Exception throwing and handling was improved in DC3a. The interface was
simplified by removing the virtually-unused inheritance from {\tt
DataProperty}.  Instead, five features were added:

\begin{itemize}

\item A simple string message is used as the primary payload of the
exception, for compatibility with standard C++ and Python exceptions.
This also makes throwing an exception simpler for the programmer.

\item A combination of macros and classes was used to automatically
include the file, line, and function where the exception was thrown.
This feature improves the debuggability of the code.

\item Other macros allow arbitrary additional parameters to be added to
subclasses of the generic {\tt lsst::pex::exceptions::Exception} class.

\item Further macros were defined that allow additional traceback
information, including additional messages, to be added to a caught and
rethrown exception.

\item LSST C++ exceptions are transformed via SWIG code into instances
of the {\tt LsstCppException} class in Python, which inherits from the
standard Python {\tt Exception} class.  The underlying SWIG-wrapped C++
exception is available as an argument of the {\tt LsstCppException}, and
the C++ exception's message is automatically included as part of the
Python exception's message.  Previously, LSST C++ exceptions were
transformed into Python exception classes that did not inherit from the
standard Python {\tt Exception} class.

\end{itemize}

The result was an exception design that was simpler to throw and that
targeted the location of the exception much more precisely.  Programmers
are still not using the more advanced features of the exception facility
such as re-throwing or additional parameters; if they are found to be
unneeded, the class can be further simplified in the future by removing
them.


\subsubsubsection{Provenance}

The Orca orchestration layer (see Section
\ref{sec:PipelineOrchestration}) in DC3a generates enhanced provenance
information.  In particular, the software environment and the contents
of the policy files used to run the production are written both to log
files and to the database.  Recording the software environment allows
the exact software configuration used for the run to be reproduced
later, while recording the policy files captures both platform
configuration information such as the compute nodes and database used as
well as all configurable science algorithm settings.

This provenance information, in combination with an event sent to the
production, is sufficient to enable accurate reconstruction of a given
data product resulting from that event, although a demonstration of
automated reconstruction was deferred.  When combined with the full
sequence of events sent to the production, the provenance allows exact
duplication of a given run.  The recorded provenance proved highly
useful while debugging algorithmic issues since it simplified the
construction of small reproducible test cases demonstrating problems.

The software environment is characterized by the versions of packages
maintained by {\tt eups} that are ``setup'' at the time of production
execution.  In addition, the actual directories declared as the
installation locations of the packages are also persisted, allowing
locally-setup packages and packages installed under {\tt LSST\_DEVEL} to
be identified.

The recorded policy file information includes the contents on a per-key
basis as well as an MD5 hash of the file contents and the file's
last-modified-time.  These latter two items are intended for eventual
use to remove duplicate entries when policy files are reused across
multiple runs.

The provenance written to the database goes into two sets of tables: one
set in the per-run database and one in a global database ({\tt
DC3a\_DB}) that spans all DC3a runs.  The global database permits
queries to find runs that used a given configuration.  For example, this
query finds all runs that had the {\tt pixelScaleRangeFactor} set to a
number other than 1.1:

\begin{verbatim}
SELECT prv_Run.runId, prv_PolicyKey.keyName, prv_cnf_PolicyKey.value
FROM prv_Run, prv_PolicyKey, prv_cnf_PolicyKey
WHERE prv_PolicyKey.policyKeyId = prv_cnf_PolicyKey.policyKeyId
  AND keyName = 'pixelScaleRangeFactor'
  AND value != '1.1'
  AND FLOOR(prv_PolicyKey.policyFileId / 65536) = prv_Run.offset;
\end{verbatim}

Similarly, this query finds all runs that used version 3.0.9 of the {\it
meas\_algorithms} package:

\begin{verbatim}
SELECT runId
FROM prv_SoftwarePackage NATURAL JOIN prv_cnf_SoftwarePackage
     JOIN prv_Run ON (FLOOR(prv_SoftwarePackage.packageId / 65536) = offset)
WHERE packageName = 'meas_algorithms' AND version = '3.0.9';
\end{verbatim}



% section 5.1.3: Pipeline Orchestration

\subsection{Processing Orchestration and Control}

Closely related to pipeline execution packages are the control
packages ({\tt ctrl\_}).  The {\tt ctrl\_events} package provides the
event framework that allows pipelines to talk to each other.  The {\tt
ctrl\_evmon} provides the event monitor services which can listen to
events (particularly logging events) and react intelligently.  The
{\tt ctrl\_orca} realizes the {\it orchestration layer} responsible
for launching pipelines.  An important part of this layer must do is
record {\it provenance} information--all of the data describing the
software environment and policy paramters that was used to configure
and execute a pipeline.  

\subsubsubsection{Pipeline Orchestration} \label{sec:PipelineOrchestration}

The orchestration layer is a collection of Python objects used to set up and
launch pipelines. This collection of objects is invoked via a command line
utility named \texttt{orca.py}.

\texttt{Orca} takes as arguments a run ID and a policy file. The run ID is used
to identify the pipeline run. The policy file lists configuration information
about the platforms where pipelines are to be launched, the type of database to
use, and the event broker to use.

\texttt{Orca}'s object hierarchy is architected so that platform specific work
is handled by subclasses and more generalized work is handled by super-classes.
The objects that \texttt{Orca} instantiates and uses are specified by policy
files. As we add support for new platforms and database configurations, we can
expand this object hierarchy, customizing platform specific tasks to the methods
that will be executed without having to rework the main code for \texttt{Orca}.

The three main objects that \texttt{Orca} uses are the
\texttt{ProductionRunManager}, the \texttt{DatabaseConfig\-urator} and the
\texttt{PipelineManager}.  The \texttt{DatabaseConfigurator} and the
\texttt{PipelineManager} objects are subclassed to provide database and platform
specific functions.

The \texttt{ProductionRunManager} looks up in the policy the type of
\texttt{DatabaseConfigurator} to use and instantiates the appropriate object.
The \texttt{DatabaseConfigurator} is responsible for looking up the appropriate
database authentication information for the user, connecting to the database,
and creating the necessary database tables that will be used to record
information about the run.

The \texttt{ProductionRunManager} then creates a \texttt{Provenance} object used
to record information to the database about the provenance about the run. This
provenance information includes all information in the command line policy file.

The \texttt{ProductionRunManager} uses the policy to lookup and create platform
specific \texttt{Pipeline\-Manager} objects to configure and run pipelines. The
pipeline policy and software environment provenance is recorded for each
pipeline to the database.

The \texttt{PipelineManager} that is invoked is specified in the platform
section of the policy for that pipeline.  This section describes the root of the
pipeline directories, the names of directories to create, and the pattern to use
to create those directories.  It also describes the hardware configuration to
use, and the nodes available for use by a pipeline. The \texttt{PipelineManager}
places all the files required for the runs in locations where the pipelines can
get to them.

Once all the policy and data files are put into place, the Production run
manager has each \texttt{PipelineManager} launch its pipeline.

\subsubsubsection{Provenance}

The Orca orchestration layer (see Section
\ref{sec:PipelineOrchestration}) in DC3a generates enhanced provenance
information.  In particular, the software environment and the contents
of the policy files used to run the production are written both to log
files and to the database.  Recording the software environment allows
the exact configuration of LSST-packaged software used for the run to be
reproduced later, while recording the policy files captures both
platform configuration information such as the compute nodes and
database used as well as all configurable science algorithm settings.

This provenance information, in combination with an event sent to the
production, is sufficient to enable accurate reconstruction of a given
data product resulting from that event, although a demonstration of
automated reconstruction was deferred.  When combined with the full
sequence of events sent to the production, the provenance allows exact
duplication of a given run.  The recorded provenance proved highly
useful while debugging algorithmic issues since it simplified the
construction of small reproducible test cases demonstrating problems.
\iffalse
\RHL{My impression was that we didn't yet have the ability to 
down the stage configuration associated with an error, or
is this what, ``a demonstration of 
automated reconstruction was deferred''?}
KTL - We can get the configuration easily.  What is not so readily
available is the inputs to a given stage, many of which may not be
pipeline data products (they are just internal).  This is why the text
says "simplified" and not "automatically enabled".  The clipboard inputs
can always be re-created by re-running the pipeline, but this may be
inconvenient and inefficient.
\fi

The software environment is characterized by the versions of packages
maintained by {\tt eups} that are ``setup'' at the time of production
execution.  In addition, the actual directories declared as the
installation locations of the packages are also persisted, allowing
locally-setup packages and packages installed under {\tt \$LSST\_DEVEL} to
be identified.

The recorded policy file information includes the contents on a per-key
basis as well as an MD5 hash of the file contents and the file's
last-modified-time.  These latter two items are intended for eventual
use to remove duplicate entries when policy files are reused across
multiple runs.

The provenance written to the database goes into two sets of tables: one
set in the per-run database and one in a global database ({\tt
DC3a\_DB}) that spans all DC3a runs.  The global database permits
queries to find runs that used a given configuration.  For example, this
query finds all runs that had the {\tt pixelScaleRangeFactor} set to a
number other than 1.1:

\begin{verbatim}
SELECT prv_Run.runId, prv_PolicyKey.keyName, prv_cnf_PolicyKey.value
FROM prv_Run, prv_PolicyKey, prv_cnf_PolicyKey
WHERE prv_PolicyKey.policyKeyId = prv_cnf_PolicyKey.policyKeyId
  AND keyName = 'pixelScaleRangeFactor'
  AND value != '1.1'
  AND FLOOR(prv_PolicyKey.policyFileId / 65536) = prv_Run.offset;
\end{verbatim}

Similarly, this query finds all runs that used version 3.0.9 of the {\it
meas\_algorithms} package:

\begin{verbatim}
SELECT runId
FROM prv_SoftwarePackage NATURAL JOIN prv_cnf_SoftwarePackage
     JOIN prv_Run ON (FLOOR(prv_SoftwarePackage.packageId / 65536) = offset)
WHERE packageName = 'meas_algorithms' AND version = '3.0.9';
\end{verbatim}



% section 5.2: Database Schema Modifications

\subsection{Database Schema Modifications since DC2}


The MOPS schema was integrated into the master copy of the
LSST baseline schema in Enterprise Architect. The 
integration included realigning names, units, and 
merging tables where appropriate.


Amplifier-level exposure metadata was introduced,
in addition to the CCD-level and FPA-level metadata 
already present in DC2.


SDQA-related schema was introduced, and this is 
covered in detail in Subsection~\ref{sdqasubsection}.


Schema for capturing provenance related to policies
was introduced.


The whole schema was refactored to clean up types
and realign it with LSST naming conventions.


The schema for the Object, Source and DIASource tables 
was modified to reflect the new needs of the 
application classes for astronomical-object transients.


\subsubsection {Schema browser}

A schema browser was built and deployed; the URL is
http://dev.lsstcorp.org/schema/. This graphical, web-based
browser allows a quick, easy and convenient way to look
at the database-schema structure, including browsing tables, 
column names, their types, units, UCDs, and descriptions.
Descriptions and units for most columns were added, but
the UCD fields remain empty.

This work included expanding the master copy of the schema 
in Enterprise Architect to capture information 
about per-column units, UCDs and descriptions.
A script was written to translate this 
information into a format readable by the schema browser. 
This allows us to manage all the information about the schema 
in one central place (EA), in a database-agnostic form.





% % section 5.3: Operational Issues

\subsection{Operational Issues}


Database authorization was reworked. It was non-existent in DC2:
everybody was allowed to do anything. In DC3a we tighten
authorization: introduced scheme where each user had her/his
own ``namespace'': each user is granted full access to
her/his databases, and read-only access to other databases.
An administrative tool for granting lsst-specific
privileges for new users was written.


We also introduced metadata about runs. This allows us
to easily related databases with runs, and cleanup old
or obsolete runs based on different criteria such as
creation time or run owner.


Database setup scripts were reworked. A thin layer between
the database and the admin scripts was introduced to contain
the code directly interacting with the database. This will
simplify migration to other database technology, should
this become necessary in the future. The database setup
scripts were also integrated with policies.






\subsection{Application Framework}

Most of the additions to the applications framework
classes are discussed elsewhere (e.g. \code{Statistics}, \Sec{secBackground}; \code{Psf}, \Sec{secPsf}),
but the most far-reaching of the changes, the new image API, deserves its own section.

\subsubsection{The Image Classes}
\label{secImageClasses}

In DC2 pixels in images and masks were managed using a toolkit from NASA-Ames, the ``Vision Workbench'' (\code{vw}).
Access to the pixels in an image was achieved by explicitly following a pointer to a \code{vw} class,
thereby exposing all of \code{vw} API, including any incompatible changes that might appear in
the future.

A further major problem with \code{vw} is that its concept of pixel access is restricted to a simple
STL-like iterator, whereas many astronomical algorithms require examination of the neighbourhood of
a pixel.

At the end of DC2 we therefore decided to rewrite all of our image classes to use \code{boost::gil}
(an image library originating at Adobe) which has a richer set of ways of accessing pixels;  in
order to preserve encapsulation, we added a thin wrapper layer above the raw \code{boost::gil} calls.
Another major goal of the rewrite was to make the code that manipulated the pixels of a single
\code{Image} identical to that that manipulates a \code{MaskedImage}'s pixels --- thereby enabling
generic programming of e.g. convolutions.

All of these goals have been achieved for DC3a.  The new APIs have proved to be convenient and
powerful, and the code generated is efficient,\footnote{at least with g++ versions $>= 4.2$} even when
manipulating images, masks, and variances simultaneously when processing \code{MaskedImage}s.



% -- Section 6
\section{Application Layer} \label{sec:applayer}

% Section 6.1: ISR

\clearpage
\subsection{Instrument Signature Removal (ISR) Pipeline}

Because the vast majority of ISR tasks require trivial pixel
operations (e.g. subtraction of or division by a master calibration
image) the sub--stages were written in {\tt Python}, implemented in
the file {\tt \$IP\_ISR\_DIR/python/lsst/ip/isr/IsrStages.py}, with
access to these sub--stages available to a pipeline Stage by {\tt
import lsst.ip.isr.IsrStages as isrStages}.

Each sub--stage is given a representative string (e.g. \texttt{ISR\_TRIM})
%{\tt
%    std::string const\& ISR\_LIN   = "ISR\_LIN";    ///< Linearization           \\
%    std::string const\& ISR\_OSCAN = "ISR\_OSCAN";  ///< Overscan                \\ 
%    std::string const\& ISR\_TRIM  = "ISR\_TRIM";   ///< Trim                    \\
%    std::string const\& ISR\_BIAS  = "ISR\_BIAS";   ///< Bias                    \\
%    std::string const\& ISR\_DFLAT = "ISR\_DFLAT";  ///< Dome flat               \\
%    std::string const\& ISR\_ILLUM = "ISR\_ILLUM";  ///< Illumination correction \\
%    std::string const\& ISR\_BADP  = "ISR\_BADP";   ///< Bad pixel mask          \\
%    std::string const\& ISR\_SAT   = "ISR\_SAT";    ///< Saturated pixels        \\
%    std::string const\& ISR\_FRING = "ISR\_FRING";  ///< Fringe correction       \\
%    std::string const\& ISR\_DARK  = "ISR\_DARK";   ///< Dark correction         \\
%    std::string const\& ISR\_PUPIL = "ISR\_PUPIL";  ///< Pupil correction        \\
%    std::string const\& ISR\_CRREJ = "ISR\_CRREJ";  ///< Cosmic ray rejection    \\
%}
%
that is used when logging the results of the ISR processing (
e.g. {\tt lsst.ip.isr.trim DEBUG: ISR\_TRIM using trimsec
[1:512,1:2048] } )
%
as well as to provide provenance in the Exposure's Metadata 
%
(e.g. {\tt ISR\_TRIM= 'using trimsec [1:512,1:2048]; Fri Mar 27
03:33:56 2009' } ).
%
Each sub--stage also assigned a bitplane to record the type of
processing done to each image.  This information would typically be
checked for before undertaking a given sub--stage, so as to not repeat
processing steps.
\RHL{Is this a bit in a status longword or a bit in a \texttt{Mask}? If the
latter, how is it used?}

%{\tt 
%    enum StageId {                                        \\
%        ISR\_LINid   = 0x1,   ///< Linearization           \\
%        ISR\_OSCANid = 0x2,   ///< Overscan                \\
%        ISR\_TRIMid  = 0x4,   ///< Trim                    \\
%        ISR\_BIASid  = 0x8,   ///< Bias                    \\
%        ISR\_DFLATid = 0x10,  ///< Dome flat               \\
%        ISR\_ILLUMid = 0x20,  ///< Illumination correction \\
%        ISR\_BADPid  = 0x40,  ///< Bad pixel mask          \\
%        ISR\_SATid   = 0x80,  ///< Saturated pixels        \\
%        ISR\_FRINid  = 0x100, ///< Fringe correction       \\ 
%        ISR\_DARKid  = 0x200, ///< Dark correction         \\
%        ISR\_PUPILid = 0x400, ///< Pupil correction        \\
%        ISR\_CRREJid = 0x800, ///< Cosmic ray rejection    \\
%    };                                                    \\
%}


\subsubsection{ISR Tasks Implemented for DC3a}

The sub--stages that were implemented for DC3a, using their subroutine
names and listed in the order they are called by the {\tt process()}
method of the main ISR stage, are :

\begin{itemize}

\item ExposureFromInputData : This assembles an Exposure from the
input science Image, Metadata, and Bounding Box of the Amplifier
within the CCD.  A zero--valued Mask is created, and the variance
Image is synthesized from the science pixels and the {\tt Gain} from
the Metadata.  These are combined into a MaskedImage.  A WCS is
synthesized from the input Metadata, and finally an Exposure assembled.
\RHL{What is this WCS used for?  Is it intended to be generated from the
boresight and a focal plane model, and therefore fair game for injection
into the WCS calculation?}

\item LookupTableFromPolicy : Creates a linearization lookup table
({\tt LookupTableMultiplicative} or {\tt LookupTableReplace}) from an
input Policy.
\RHL{What's the distinction being drawn between these two?}
For DC3a, we create a replacement lookup table that
merely replaces a pixel value by itself, since we don't know the true
non--linearity of the CFHT data, and the Simulated data are linear.

\item Linearization : Apply the lookup table generated above to the
science image.  The location of this sub--stage within the overall ISR
processing depends on the details of how the linearity curve was
determined.  For example, it might go after bias and dark subtraction,
if the linearity calibration frames were themselves bias and dark
subtracted before analysis.
\RHL{If there are non-linearities in different places in the signal
chain, we could even need two corrections in the long run}

\item SaturationCorrection : The saturation keyword is retrieved from
the Exposure's Metadata, and the Detection algorithm is called to find
all pixel values equal or greater than this value, returning a list of
Footprints.  If the Policy contains an option to grow these
Footprints, they are isotropically grown.  The associated pixels in
the Mask have their saturated bit set.  The Policy also determines
whether or not to interpolate over these pixels - if interpolation is
requested, this functionality is called and additional bits are set in
the Mask indicating the pixels were interpolated.  A default Psf with
full--width--half--maximum (FWHM) of 5 pixels is used in the
interpolation.
\RHL{We need to revisit this 5 pixels;  I may have chosen it, but it
seems too large.}

\item OverscanCorrection : The overscan region is retrieved from the
Exposure's Metadata, and a subExposure created using this bounding
box.  Currently, the user can chose to subtract the mean or median of
all pixels in this overscan region from the image.  

\item TrimNew : The trim section is retrieved from the Exposure's
Metadata, and a new subExposure is created containing the trimmed
science Exposure.  The pixel origin is shifted accordingly, and the
trim section removed from the Exposure's Metadata.  The new Exposure
is returned.

\item BiasCorrection : The master bias Exposure is read from the
Clipboard, and subtracted from the science Exposure.

\item DarkCorrection : The master dark Exposure is read from the
Clipboard.  The dark Exposure is scaled to the science Exposure's
integration time, and then subtracted from the science Exposure.

\item FlatCorrection : The master flat Exposure is read from the
Clipboard.  The flat Exposure is scaled by its mean or median, and
divided out of the science Exposure.

\iffalse
\item IlluminationCorrection : This is the same functionality as {\tt
FlatCorrection}.   \RHL{...but may need a different flat, depending on
how we handle the photometric v. cosmetic flats.}
\fi

\item MaskBadPixelsDef : A list of instrumental pixel \texttt{Defects} is read 
from the input Clipboard.  The corresponding bits are set in the
Exposure Mask.  The Policy also determines whether or not to
interpolate over these pixels - if interpolation is requested, this
functionality is called and additional bits are set in the Mask
indicating the pixels were interpolated.  A default Psf with a FWHM of
5 pixels is used in the interpolation.
\RHL{Again, we need to reconsider this ``5''}
\RHL{Do we need to say where these come from?  We should say that they
are not complete for DC3a}

\item CrRejection : A background model is generated for the Exposure, 
and subtracted off of the science Image.  A default Psf with a FWHM of
5 pixels is used to compare detected Sources; Sources sharper than the
Psf are masked as cosmic rays, and interpolated over.
\RHL{Again$^2$, we need to reconsider this ``5''}

\item CalculateSdqaRatings : Two SdqaRatings are generated by the ISR : 
{\tt ip.isr.numSaturatedPixels} and {\tt ip.isr.numCosmicRayPixels}.
Both of these are synthesized from the final Mask by counting the
number of pixels with the {\tt SAT} and {\tt CR} bits set,
respectively.
\RHL{Wouldn't it be better to return the number of CRs?  This is
available from the CrRejection code}

\end{itemize}


\subsubsection{Results}

The only truly testable portion of the ISR during DC3a was the
identification and masking of the cosmic rays synthesized for the
exposureId = 1 images.  \XXX{Jeonghee is looking at this.}

\subsubsection{Issues}

Several issues both minor and outstanding were raised during
development of the ISR for DC3a.  These include :

\begin{itemize}

\item The need for a standardized set of Metadata drove the need for the 
establishment of a {\tt datatypePolicy} for each input dataset (CFHT
and Sim), containing a mapping to the {\tt dc3MetadataPolicy}
established for DC3a.  In practice, this takes input FITS header
keywords and maps them to the appropriate Metadata key that is
expected to be extracted from a database query.  While this will be
needed for {\it any} input dataset we run through our pipelines, we
should reevaluate the implementation of this post--DC3a.

\item The CFHT data sections (e.g. {\tt TRIMSEC, BIASSEC}) are stored
in FITS 1--indexed convention, with the final index inclusive.  For
example, {\tt BSECA = '[1:32,1:4644]'} indicates that a bias section
runs from the first to thirty-second pixel on the x-axis.  As LSST is
using 0--indexed pixel addressing convention with the final index
inclusive, this should read {\tt BSECA = '[0:31,0:4643]'}.  This
adjustment is currently made when turning the input data section into
a {\tt lsst::afw::image::BBox} in {\tt isr::BBoxFromDatasec}; however
in the future we need to be careful to standardize the subExposure
formatting.
\RHL{We really need a proper description of the CFHT (or any other)
camera independent of the FITS header keywords;  at this point the
fortran-based indexing will be irrelevant}

\item When the background model is fitted for and 
subtracted is still an open issue.  Currently this is utilized at the
CrRejection stage; however the image returned by the Isr has the
background added back in.  Ideally the background should be fitted for
and subtracted once.
\XXX{Yes and no.  For difference imaging the desideratum is to
make the difference image flat;  the subtraction for CR removal
is needed by the algorithm and is convenient (but we could achieve
this in the CR code without actually subtracting).  For coadds
and deep detection we don't want to subtract the background this
early --- we do need to match it between exposures.  This is important
to get large, extended, astrophysical structures right}

\item A formal calibration products database is needed.  The workaround 
utilized for DC3a (text file on disk) is not a long--term solution.

\item A class representing attributes of each Detector is also needed.  
This should contain information on e.g. its gain, readnoise, and
defect list.  For DC3a this was solved by examining the FITS headers of
each input file for the relevant Metadata.  Ideally this will instead
happen through a database query constructing an instance of this
Detector class.
\RHL{Some of this is covered
by Andy Rasmunssen's proposal for describing the focal plane, but
it needs to be augmented with e.g. location of prescan/extended serial/postscan
pixels, gains, readnoises, bad pixels (with classification) etc.  If we do
this right, we'll resolve most of these `issues'.}


\end{itemize}

\subsubsection{ISR Tasks Post-DC3a}

The sub--stages (or features of the sub--stages) that remain to be
implemented are :

\begin{itemize}

\item The details of how saturated pixel Footprints are grown.
\RHL{Do we want to discuss the reasons for growing the saturation mask?
You need to grow up and down, but if the serial's got a high enough
full well and CTE you may not need to grow `sideways'.  For most
devices the profile of the top of the bleed trail is complicated,
and you may have to be careful to set the level not to be fooled}

\item Allow functional fitting of the overscan region.  Currently a
single number is calculated for the overscan value (the median of all
overscan pixels) and subtracted from the entire image.  However, the
overscan may vary, and typically a functional form (spline, Legendre
polynomial) of a given order is fit to the data.  The type of fitting
({\tt mean, median, Legendre}) will be controlled by the input Policy.
\RHL{Is the ``overscan'' all real overscan, or does it include an
extended register?  If the latter, we need a config file to specify
which pixels to use and which to ignore}.

\item Fringe frame correction.  The amplitude of the fringes in a
given science image needs to be fit for, and a scaled version of the
master Fringe frame subtracted from the science image.
\RHL{We may well need more than one fringe frame (for different
atmospheric components), and hope to use information from the
atmospheric monitor.  Not that this helps with CFTH data...}

\item Cross--talk correction.  This will happen at the camera in
Nightly Processing, but will have to be redone by the ISR for Data
Release Processing.  This is an inherently non--parallel task,
requiring at the least all Amplifier images from a given CCD.
\RHL{Do we even know this?}

\item Creation of the master calibration products, including bias,
darks, flats, illumination correction images, pupil correction,
linearization tables, bad pixel masks, and cross--talk correction
matrices.  Each of these requires a specialized set of input
calibration data that was out of scope for DC3a.

\end{itemize}


% Section 6.2: Background Subtraction

\subsection{Background Subtraction}
\label{secBackground}

% intro 

A small number of new classes were developed for the {\tt afw} package
to perform photometric background estimation.  A \code{Statistics}
class was written to compute a variety of standard statistics (mean,
median, standard deviation, etc.) for the pixel values in an
\code{Image} or \code{MaskedImage}.  The classes \code{Interpolate}
(base) and \code{SplineInterpolate} (derived) were written to provide
code to perform a natural spline interpolation over $x,y$ pairs, 
and a \code{Background} class, which uses the \code{Statistics} and
\code{Interpolate} classes, was written to estimate the photometric
background for use in sky subtraction.

\subsubsection{Algorithm Description}

The basic assumption in the algorithm is that an image is {\it
relatively} sparsely populated with astrophysical sources, and that the
majority of the pixel values are random variates about the local mean
sky.

% break up the image into a grid of sub-regions
% get the 3-sigma clipped median in each sub-region (cite NR for fast median)
% --> these values should be rough approximations to the sky in each grid cell

To estimate local sky levels, an image is first subdivided into a grid
of subimages.  In each grid cell, a
3$\sigma$-clipped\footnote{Clipping is symmetric, and thus unbiased 
for a symmetric distribution such as a Gaussian.} mean is computed to
represent the sky level.  Sky levels will often be contaminated with
stellar flux from objects which are below the detection threshold, and
a median would be a more robust measure of the central sky value under
these circumstances.  However, as such sub-threshold sources will also
contribute to the detected stellar PSFs, their biasing influence is
statistically desirable in an estimate of the background as they are a
part of the background.  For this reason, a clipped mean is chosen
over a median.  The clipped mean for each cell is assigned to the
row,column coordinate of the cell's central pixel. 

% use a bicubic spline to construct a background image (site NR for bicubic spline)
% - spline along each 'column' of medians to interpolate all pixels in the column
% - spline across each row using the values from the columns

The resulting grid of clipped means is interpolated to create a
background image with the same dimensions as the original.  A bicubic
spline interpolation algorithm is used to mesh the grid smoothly over
the image.  For our bicubic interpolation, the
grid values are first splined along the columns to produce smooth
pixel values along each grid column.  The full background image is
then produced by splining the interpolated column values across each
row.  The first interpolation (along grid columns) is far less
computationally intensive than the second (across {\it all}
rows).  It would be equally valid to perform the first interpolation
across the grid rows and the second along all columns.  However, the
row-major (i.e. natural C/C++) structure of the LSST images made it
advantageous to adopt the former option.

\subsubsection{Assessment}

The background estimation code was successfully tested on images with
known background levels: (1) all pixels set to a constant value, (2)
pixel values increasing linearly across the frame, and (3) an image of
noise having known statistical properties.  The tests are included
with the source code.

% not-a-knot

The {\it natural}\footnote{i.e. vanishing second derivative at the end
knots} spline algorithm used for interpolation can suffer from
ringing near the edges, and it is our intention to continue
development by implementing a more robust {\it not-a-knot} spline in
its place.


% Section 6.3: Source Detection

\subsection{Source Detection}



% Section 6.4: PSF Determination

\subsection{PSF Determination}
\label{secPsf}

\subsubsection{Classes}

The PSF virtual base class, \code{Psf}, is currently in the
\code{lsst::meas::algorithms} package, but should be moved
to \code{lsst::afw} for DC3b.

\code{Psf} supports a PSF that is a function of position.  
In addition to constructors and destructors, the class interface
permits convolving an image with the PSF or evaluating a desired
pixel (or returning an \code{Image}) of the PSF at a given position in a frame.

There are two concrete \code{Psf} classes in DC3a: \code{dgPsf} (a circularly-symmetric double Gaussian) and
\code{pcaPsf} (a linear combination of basis functions, modelled after SDSS's PSF representation).  Both of these
classes are instantiated using a helper class, \code{PsfFactory}, that takes a string argument to identify the desired
type of PSF; this enables us to add a new PSF representation without modifying the base class.\footnote{In fact, the new
  PSF class could be separately compiled and loaded at run time using python's \code{import} command.}

\subsubsection{Spatial Modelling}

In modelling a PSF's spatial structure, we need to ensure that the PSF is well modelled at all
points in the image, rather than just where we happen to have bright isolated stars.  It is
therefore important to use stars which are as uniformly as possible distributed, even if this
means that we have to accept fainter stars than we might like.

A given image is divided up into cells, with each cell represented by an instance of \code{SpatialCell}. These cells
each contribute only a limited number of objects to the spatial fit.  The \code{SpatialCellSet} class maintains all of
these cells as an STL collection of \code{SpatialCell}s.  Each cell itself contains an STL collection of instances of
classes derived from \code{SpatialCellCandidate} (e.g. the PSF stars themselves).

\code{SpatialCellSet} uses the Visitor pattern \citep{GoF} to allow the user to apply a functor to the `best'
\code{SpatialCellCandidate} in each \code{SpatialCell}, estimating a spatial model.  The user can then go through the
candidate lists rejecting or approving each object; if needs be the spatial model can then be improved.  If all
instances in a list are rejected from the spatial model, the least-bad one will be used.

A superclass of \code{SpatialCellCandidate}, \code{SpatialCellImageCandidate}, contains an image
of some sort; in the case of PSF determination, this is candidate PSF star, but the
same framework could be used in \code{ip\_diffim} in which case the image would
be a difference image kernel.


\subsubsection{PSFs as sums of PCA components; \code{pcaPsf}}

The DC3a PSF is based on a \code{lsst::afw::math::LinearCombinationKernel}, where
the components are the eigen-images derived from a set of stars in the image (cf. \citet{photoADASS});
we are not wedded to this representation post-DC3a.

In terms of the previous section,  we first find a set of \code{Source}s in an image
that may be good isolated stars,  and insert them into a \code{SpatialCellSet}.  The
routine \code{createKernelFromPsfCandidates} visits each cell and performs the PCA
to find the desired eigen-images using a Visitor and the \code{lsst::afw::image::ImagePca} class.
Given this kernel, \code{fitSpatialKernelFromPsfCandidates} uses a different
Visitor to evaluate the $\chi^2$ of the fit, and then \code{minuit} to find
the best-fit spatial variation (modelled in terms of a 2-dimensional polynomial).  Finally,
we call the \code{createPSF} factory function to generate a \code{Psf}.

In semi-pseudo-code this looks like:
\begin{center}
  \begin{minipage}{13cm}
    \small
\begin{verbatim}
mi = MaskedImageF()
cellSet = SpatialCellSet(BBox(PointI(0, 0), width, height), 100)

for source in sourceList:
    cellSet.insertCandidate(makePsfCandidate(source, mi))

kernel, eigenValues = \
       createKernelFromPsfCandidates(cellSet, nEigenComponents, spatialOrder,
                                     kernelSize, nStarPerCell)

fitSpatialKernelFromPsfCandidates(kernel, cellSet, nStarPerCellSpatialFit, tol)

psf = createPSF("PCA", kernel)
\end{verbatim}
  \end{minipage}
\end{center}
We have omitted the rejection of bad candidates in the interest of clarity.


\subsection{Source Measurement}

Source measurement occurs only in the context of an image, thus 
\code{MeasureSources} - a set of centroid, shape, and photometry measurement 
algorithms - is constructed from an \code{Exposure}; it is templated over 
\code{Exposure}'s pixel type. \code{MeasureSources}  acts as the driver behind
the generalized \code{Footprint} iterator \code{FootprintFunctor} in applying 
its set of algorithms to a \code{Footprint} over the \code{Exposure}. 
Which algorithms to use can be specified by name in a \code{Policy} file.



% Section 6.5: WCS Estimation

\subsection{WCS Determination}

\RHL{Fergal needs to write this}



% Section 6.6: Image Subtraction

\subsection{Image Subtraction}

\subsubsection{Implementation in Framework}

Two major framework ({\tt afw}) improvements facilitated the
implementation of {\tt ip\_diffim} for DC3a, both coming about due to
the removal of {\tt Vision Workbench} from the software stack.  The
first was the replacement of the {\tt VW} pixel access operations,
while the second was the adoption of the {\tt Eigen} package for
linear algebra operations.  

The Applications Framework ({\tt afw}) release version 3.1 included the first stable
release of the new pixel accessor and iterator functionality.  The entirety of the  {\tt ip\_diffim} 
C++ code base had to be updated to use this new API.  

The {\tt Eigen} matrix math library was adopted due to its speed and templated nature.  Compared to the {\tt VW} matrix
math operations, its faster



\subsubsection{Speed Ups}

Specialized convolution by delta--function kernels, as well as overall caching improvement in the convolution.  Eigen
is also way faster.

\subsubsection{Algorithm Improvements}

The major improvement in the {\tt ip_diffim} package over DC2 was in designing an infrastructure for
the spatial modelling of the Kernel.  The SpatialCell class divides the image up into a grid, and attempts
to build a single Kernel model for each Cell by choosing the brightest object to build a Kernel around.  
If that model turns out to be inadequate (i.e. the resulting difference image is noisier than expected), 
another object is used in the modelling.  This process is repeated until a good Kernel is found, or all objects are 
deemed to be inadequate, in which case the Cell is considered ``empty''.  In this way we generate
an even grid of constraints of the Kernel model across the image, which will facitlitate the creation of the spatial Kernel model.
The spatial fitting is now itegrated closely with the SpatialCell class, iterating over all Cells and using their respective
models as constraints on the Kernel.  


Growing the Footprints by the right size.

Decrease the number of eigen Kernels.

\subsubsection{Results}

\subsubsection{Future Work}

Setting the Kernel size based on the Psf FWHM.

Further investigation of the convolution vs. deconvolution kernel issue is warranted.  Using the {\tt SimDeep} dataset, we
have determined that the derived Kernels are far noisier than expected even when the process is known to be a smoothing convolution.
The Kernels are far less self--similar than expected; consequently their spatial modelling will be difficult.

The SpatialCell functionality was initially written for fitting of the {\tt ip\_diffim} convolution Kernel.  However, it was subsequetly rewritten in a more advanced
form for modelling of the spatially varying Point Spread Function.  A task post--DC3a is to update the {\tt ip\_diffim} to use this
newer SpatialCell functionality.


% Section 6.7: MOPS

\subsection{MOPS}

The version of NightMOPS developed for DC3a presents a number of significant 
upgrades over DC2.

The basic algorithm did not change. It still involves
\begin{itemize}
    \item Identifying which known MovingObject orbits might be intersecting a 
          given field of view at a given time.
    \item Computing precise positions for those MovingObjects at that time.
    \item Passing those accurate positions, with errors, to the Association 
          pipeline.
\end{itemize}

However for DC3a
\begin{itemize}
    \item The identification of candidate MovingObjects uses KD-Tree searches 
          implemented by the Auton FieldProximity code instead of simple 
          quadratic interpolation of the orbit with a boundary check.
    \item The computation of precise ephemeris is done using JPL Horizons 
          software, the golden standard for this type of computations. 
          Previously, positions were computed using quadratic interpolation 
          between pre-computed midnight ephemeris.
\end{itemize}
These changes have the effect of improving the scientific quality of the results
and the accuracy and precision of the predicted position and associated errors.

A newer version of known Solar System orbits, produced by the PanSTARRS MOPS 
team using observations provided by the Minor Planet Center, was used to
\begin{itemize}
    \item Provide the initial catalog of known MovingObjects.
    \item Pre-compute midnight ephemeris used by FieldProximity.
\end{itemize}











% Section 6.8: Association

\subsection{Association}



% Section 6.9: Science Data Quality Analysis (SDQA) Subsystem

\subsection{\label{sdqasubsection}Science Data Quality Analysis (SDQA) Subsystem}

Prior to DC3a, work on the high-level requirements and operations concept was performed 
by Deborah Levine and Vince Mannings.  For DC3a, the SDQA development team welcomed new
members Peregrine McGehee, Russ Laher, Jeoghee Rho, and Dick Shaw.  An oral paper which
gives a succinct description of the basic design that emerged from our early efforts in 
DC3a was presented at the 2008 ADASS conference held at Quebec City \citep{laher08}\footnote{http://spider.ipac.caltech.edu/staff/laher/lsst/O7A2.pdf}.

DC3a marks the beginning of creating and testing prototype software for the SDQA subsystem, and checking it into the LSST source-code base.

The subsections that follow detail our objectives for DC3a and gauge our progress in
terms of these objectives.


\subsubsection{Objectives}

   The objectives for development of the SDQA subsystem in DC3a are as follows:

\begin{enumerate} 
\item{Develop a UML model for SDQA and integrate it into the overall LSST UML model.} 
\item{Develop a scheme for naming SDQA metrics. }
\item{Design and implement a first version of the database schema for SDQA.}
\item{Develop and test C++/Python code for SDQA data-container classes.}
\item{Develop and test a formatter C++ class for persisting SDQA data.}
\item{Integrate into LSST image-processing pipelines code that creates SDQA objects
for one or more SDQA metrics and persists the associated SDQA data.}
\end{enumerate}

\subsubsection{Accomplishments vs. Objectives}

Significant progress was made on Objective \#1.  SDQA domain model, use cases, 
and robustness diagrams were created and refined in LSST's Enterprise Architect 
software.  Work was done to integrate SDQA UML elements into the overall LSST
UML model.  Defined were the following basic SDQA objects:

\begin{description}
\item{\it SDQA Metric.}
These are diverse, predefined measures that characterize image-data quality; 
e.g., image statistics, astrometric and photometric figures of merit and associated 
errors, counts of various things, like extracted sources, etc.  Attributes of the
SDQA Metric class include name, physical units, and definition.  
\item{\it SDQA Rating.}
An SDQA Rating is 
the computed or derived value of an SDQA Metric and its uncertainty for a specific
image or image data set.  
\item{\it SDQA Threshold.}
An SDQA Threshold defines for a given SDQA Metric the 
upper and lower limits of SDQA-Rating values that indicate acceptable image-data 
quality.  \item{\it SDQA Image Status.}
An SDQA Image Status is the overall quality tag assigned to an image
after processing by the automated SDQA subsystem embedded in the image-processing
pipeline; attributes of SDQA Image Status include descriptive moniker and definition.
\end{description}

Objective \#2 was accomplished to the extent that is possible this early in the 
project.  The results of this substantial effort were written up on a Trac 
page\footnote{http://lsstdev.ncsa.uiuc.edu/trac/wiki/MetricsForSDQA}, which has 
enabled a wide-audience open discussion of the SDQA Metrics under consideration.
Additional metrics will be defined for DC3b that are informed by quantities of interest 
gleaned from our experience with the DC3a pipeline processing.

We designed and implemented a first version of the database schema for SDQA in fulfillment
of Objective \#3.  Figure~\ref{DB} shows the portion of our SDQA database-schema design
that associates SDQA Ratings with amplifier-level images.  
The primary advantage of this schema design is that persisting additional SDQA Metrics
does not require a schema change; instead you simply add new records to the SDQA\_Metric
and SDQA\_Threshold database tables to define the new SDQA Metrics and associated thresholds.
Our DC3a implementation also
includes tables for associating SDQA Ratings with image data at the CCD and FPA levels.
The schema's tables closely parallel the SDQA classes that we have identified in our UML (universal modeling language) design.

\begin{figure}
\epsscale{0.5}
\plotone{images/O7A2_1}
\caption{SDQA database-schema design (``F.K.'' stands for foreign key, and ``1~{\jot 24pt}~1*..'' stands for one record to many records).} 
 \label{DB}
\end{figure}

The Science\_Amp\_Exposure database table stores metadata for
processed images associated with independently read-out 
512$\times$2048-pixel portions of 
raw images (called amplifier 
segments).  The sdqa\_imageStatusId field in this table points to the grade
category determined for the image by the SDQA subsystem (although this was not populated
in DC3a, it will be done in DC3b).  
The SDQA\_Rating\_4ScienceAmpExposure database table is associated with 
the Science\_Amp\_Exposure database table in a one-to-many
relationship.  A processed image, in general, has multiple SDQA ratings, which
are computed at various pipeline stages, temporarily stored in pipeline-shared memory, and 
ultimately persisted in the database.


We accomplished Objectives \#4 and \#5 by developing and testing object-oriented C++
source code for the following classes:

\begin{itemize}
\item{SdqaMetric}
\item{SdqaRating}
\item{PersistableSdqaRatingVector}
\item{SdqaThreshold}
\item{SdqaImageStatus}
\item{SdqaRatingFormatter}
\end{itemize}

\noindent
In addition, we implemented SWIG wrappers for usage of these classes by Python scripts
directly.  We verified all SDQA-related source code by creating unit tests and assuring
their successful execution.  The unit test for the SDQA Rating Formatter involved 
persisting SDQA Ratings in a test database.  

All software coding was done carefully to assure that LSST coding guidelines were followed.

Although the SdqaRatingFormatter class has basic functionality for persisting SDQA Ratings
in DbStorage, it lacks capabilities for DbTsvStorage, BoostStorage, and XmlStorage.  This
will be remedied in DC3b, along with a few other inefficiencies pointed out by K.-T. Lim.

We created a package called ``sdqa'' in the LSST SVN repository and committed the SDQA-related source code there.  Release 3.0.3 of the sdqa package was used in the 
final testing for DC3a.

The next subsection demonstrates how Objective \#6 was met.


\subsubsection{Sample DC3a SDQA Ratings}

To accomplish our DC3a objective of persisting SDQA Ratings, the image-processing pipelines
were modified to instantiate SDQA-Rating objects for various SDQA Metrics of interest in
DC3a.  A tutorial was written to give the application developers explicit instructions on 
how to persist 
SDQA Ratings\footnote{http://lsstdev.ncsa.uiuc.edu/trac/wiki/SdqaRatingTutorial}.
The following are persisted results for SDQA Ratings that were queried from database
rplante\_DC3a\_u\_rlp1188, which corresponds to the largest set of images tested for DC3a:

{\tiny
\begin{verbatim}
+----------------------------+----------+---------------------+--------------------+---------------------+---------------------+
| metricName                 | count(*) | min(metricValue)    | max(metricValue)   | avg(metricValue)    | stddev(metricValue) |
+----------------------------+----------+---------------------+--------------------+---------------------+---------------------+
| ip.diffim.kernelSum        |      558 |  -0.657514177126991 |   7.98916588029419 |     4.7838537751288 |      1.420147681719 | 
| ip.diffim.residuals        |      558 | -0.0328804743717919 | 0.0019883894744249 | -0.0021186088057241 |  0.0037378047459525 | 
| ip.isr.numCosmicRayPixels  |      558 |                   0 |                  0 |                   0 |                   0 | 
| ip.isr.numSaturatedPixels  |      558 |                   0 |               2033 |      137.8853046595 |     381.85356427703 | 
| phot.psf.numAvailStars     |      279 |                   4 |                 25 |     12.992831541219 |     4.0789632806315 | 
| phot.psf.numGoodStars      |      279 |                   4 |                 25 |     12.931899641577 |     4.0955024257056 | 
| phot.psf.spatialFitChi2    |      279 |    5.02278757095337 |   1727.95910644531 |     118.54034705316 |     199.66763559788 | 
| phot.psf.spatialLowOrdFlag |      279 |                   0 |                  0 |                   0 |                   0 | 
+----------------------------+----------+---------------------+--------------------+---------------------+---------------------+
8 rows in set (0.01 sec)
\end{verbatim}
}

\noindent
Here is the database query that was used to obtain the above results:

{\small
\begin{verbatim}
select b.metricName, count(*), min(metricValue), max(metricValue),
       avg(metricValue), stddev(metricValue) 
from sdqa_Rating_ForScienceAmpExposure a, sdqa_Metric b 
where a. sdqa_metricId=b.sdqa_metricId 
group by b.metricName;
\end{verbatim}
}

For each amplifier-level image processed, SDQA Ratings for eight different SDQA Metrics 
were persisted in the database by the pipelines.  Four SDQA ratings were computed by
the PSF pipeline, two by the ISR pipeline, and two by the difference-imaging pipeline.
That different pipelines were able to persist SDQA Ratings demonstrates the flexibility of the design.

Although demonstrating the utility of specific SDQA ratings was not an objective of DC3a (it will
be an objective for DC3b), here are some highlights of the SDQA information presented in the 
above table:

\begin{itemize}
\item{Tim Axelrod's analysis suggests that extreme values of ip.diffim.kernelSum are correlated
with bad subtractions between image and template, which are ultimately caused by bad astrometry.}
\item{Although Tim Axelrod's assessment is that ip.diffim.residuals may not be useful for identifying
bad image subtractions, the above results at least show that kernels are being generated with the
expected statistics (according to Andy Becker, these ratings should have zero mean and +/- 0.1 R.M.S).}
\item{The results for ip.isr.numCosmicRayPixels show that no cosmic rays were detected, and 
may indicate a threshold-tuning issue.}
\item{Experience has shown that there is a correlation between phot.psf.numGoodStars and the 
positional uncertainties of PSF-fitted source detections, and so it is expected that this will be
a useful diagnostic.}
\end{itemize}

The above table of results is an example of information that could be included in a 
nightly SDQA summary.  Since the SDQA ratings are stored in the database, it is easy to
construct a variety of useful queries; e.g., all phot.psf.numAvailStars values for a given night,
CCD, and filter.

Although SDQA Thresholds were not tuned for DC3a, we expect this to be a component of the
work for DC3b.


\subsubsection{Additional Accomplishments}

In our ADASS paper \citet{laher08}, we pointed out that the method of 
artificial neural networks is particularly promising for classifying images in 
terms of SDQA (see Figure~\ref{ANN}).  In late 2008 we undertook some work 
involving ANNs to demonstrate that various attributes of extracted source detections 
could be used to differentiate between variable QSOs and non-variable white dwarfs 
\citet{borne09}.  Our completeness and reliability results showed that ANNs 
are superior to the method of segmenting the two populations in conventional 
color-color plots\footnote{
http://spider.ipac.caltech.edu/staff/laher/lsst/borne\_696\_Jan09-ver2.pdf
}.

\begin{figure}
\epsscale{0.2}
\plotone{images/O7A2_2}
\caption{Artificial neural network with single-hidden-layer architecture.} 
 \label{ANN}
\end{figure}


One of IPAC's current projects is development of image-processing pipelines and data archive
for the Palomar Transient Factory (PTF), a new 12-CCD camera recently installed on the Palomar
48'' Schmidt Telescope.  This camera is now in operations and acquires up to $\approx 4,800$
CCD images per night (2K$\times$4K pixels per CCD), 
which are subsequently sent to IPAC for processing and archiving.  
While this project is separately funded, we have been able to use
it fruitfully as a test bed for prototyping LSST-SDQA-subsystem designs.  
To be clear, the utility of PTF for our LSST purposes are:

\begin{enumerate} 
\item{Validating our choices for metrics that may be useful for LSST (since PTF is a 
time-domain project with some overlapping science goals).}
\item{Technology exploration.}
\item{Exploring visualization strategies for representing very complex and 
voluminous SDQA data in a comprehendible way.}
\end{enumerate} 

We have
implemented for PTF a fully automated SDQA subsystem of the design essentially
envisioned for LSST.  The PTF image-processing pipeline now populates more than 40 different
SDQA metrics in the PTF database.  A process for thresholding SDQA ratings and determining 
SDQA statuses for processed images has been implemented as a database stored function and
integrated into the PTF image-processing pipeline.  An indispensible software tool developed 
as a result of this effort is a web-browser-based SDQA GUI, which facilitates viewing of raw and
processed images and time-series plots of SDQA ratings over a specified observing night 
(see Figures~\ref{PTFSDQAGUI1} and~\ref{PTFSDQAGUI2}).  Much of the source 
code for the PTF SDQA GUI was leveraged from Java classes developed over the last ten years at 
the Spitzer Science Center (SSC) for observation planning, image visualization, and the
Spitzer Heritage Archive interface.  The PTF SDQA GUI is a Google Web Toolkit application that
queries the PTF database for image metadata and SDQA data to display and plot.  One of our
objectives for LSST in DC3b is to further develop the SDQA GUI for LSST use, i.e., set it up to 
query and visualize LSST databases and image archives.

\begin{figure}
\epsscale{1.0}
\plotone{images/PTFSDQSGUI1}
\caption{PTF SDQA GUI screen shot showing image display from query of PTF database.} 
 \label{PTFSDQAGUI1}
\end{figure}

\begin{figure}
\epsscale{1.0}
\plotone{images/PTFSDQSGUI2}
\caption{PTF SDQA GUI screen shot showing SDQA time-series plots from query of PTF database.} 
 \label{PTFSDQAGUI2}
\end{figure}


% -- Section 7
% Section 7: Results

\section{Results}

\subsection{Summary of Runs}

\subsection{Science Data Quality}

\subsection{Timing Results}



% -- Section 8
% Section 8: Conclusions

\section{Conclusions}



% -- References
\begin{thebibliography}{}

\bibitem[Borne {\it et al.}(2009)]{borne09} Borne, K. D., R. Laher, Z. Ivezic, and N. Hamam 2009,
   Petascale Object Classification of the LSST Event Stream, 
   Poster presented at the 213th AAS Meeting, Long Beach, CA, 4-8
   January 2009.

\bibitem[Bruzual \& Charlot (2003)]{bruzChar03} Bruzual, G, and
  S. Charlot 2003, MNRAS, 344, 1000.

\bibitem[Gamma \textit{et al.}(1995)]{GoF}Erich Gamma, Richard Helm, Ralph Johnson, and John Vlissides 1995,
Elements of Reusable Object-Oriented Software.

\bibitem[Laher {\it et al.}(2008)]{laher08} Laher, Russ R., Deborah Levine, Vince Mannings, 
   Peregrine McGehee, Jeonghee Rho, Richard A. Shaw, and Jeff Kantor 2008,
   LSST Science Data Quality Analysis Subsystem Design,
   Oral presentation given at the ADASS XVIII Conference, Quebec City, Ontario, Canada, 
   2-5 November 2008.

\bibitem[Lupton et al., 2001]{photoADASS}
Lupton R.~H. et al., 2001 `The SDSS Imaging Pipelines' in
ASP Conf. Set. 238, Astronomical Data Analysis Software and Systems X.
   
\bibitem[Pier et al., 2003]{SDSSAstrom} Pier, J.R., Munn, J.A., Hindsley, R.B., Hennessy, G.S.,
Kent, S.M., Lupton, R.H., \& Ivezi\'c, \v{Z}. 2003, AJ, 125, 1559

\bibitem[Press et al., 2007]{PressNR}
  W.~H. Press,  S.~A. Teukolsky, W.~T. Vetterling, B.~P. Flannery.
  Numerical Recipes: the Art of Scientific Computing.
  2007, Cambridge University Press.
   
\end{thebibliography}



\end {document}


