% section 5.1.1: Data Access

\subsubsection{Data Access}

The data access framework was upgraded for DC3a.

A new {\tt PropertySet} class replaced the old {\tt DataProperty} class.
By simplifying the interface and unifying concepts, it helped to reduce
complexity for developers, remove dependency issues, and bring
{\tt Policy} and {\tt PropertySet} into closer alignment.  In
particular, there is no distinction between a {\tt PropertySet}
containing a single key/value pair and one containing multiple key/value
pairs.  While a {\tt DataProperty} attempted to maintain the order of
the properties within it, a {\tt PropertySet} has no defined order
except within the vector of values for a given key.  The new interface
allows hierarchical {\tt PropertySet} values to be created more easily,
using ``parent.child'' notation in the key name instead of requiring the
creation of a separate {\tt DataProperty} instance for the child.

The persistence framework saw a few additions.  New formatters for PSFs
and their underlying kernels and spatial functions were written,
allowing those classes to be persisted to Boost archive files and XML
files.  This was a non-trivial task due to the large number and
diversity of subclasses involved.  The ability to persist and restore
PSFs enhanced the debuggability of the pipeline.

The formatters for {\tt Source}s and {\tt DiaSource}s (actually {\tt
PersistableSourceVector}s and \hfil\break\texttt{PersistableDiaSourceVector}s) were
also updated to match the new contents of those classes.  These
formatters were at the heart of inter-slice communication in DC3a as
{\tt Source}s used for WCS and PSF determination were communicated via
the database so that an entire CCD's information could be used for the
calculations.

A central, consistent facility for processing {\tt LogicalLocation}
strings was provided, allowing information from an event, other {\tt
Clipboard} items, or a {\tt Policy} to be substituted into them with
optional formatting.  This feature was used extensively to generate
pathnames and database URLs.  It allowed stage policies to be
platform agnostic while the pipeline orchestration layer combined
with the pipeline harness to specify translations from
platform-independent form to platform-dependent form.

Database access was improved by implementing the {\tt DbStorage}
portability layer directly in terms of the MySQL C API instead of using
CORAL/SEAL as a second level of indirection.  This reduced the size and
complexity of the software stack without sacrificing portability.
Security was enhanced by improving the specification of database
authentication credentials and the deprecation of shared database
accounts.  The database interface now allows the limited ability to
execute raw SQL statements (without return values), allowing some
computations to be done efficiently entirely on the database server, and
it also allows the use of expressions and multi-part field names in
queries.

The {\tt DateTime} class that is used as a utility to deal with dates and
times throughout the LSST software stack had its interface significantly
revamped to improve the clarity of the timebase (UTC or TAI) used for
its arguments and results.  (The internal representation of times always
uses TAI.)  Conversions from {\tt DateTime} to and from ISO8601-format
strings were implemented.

