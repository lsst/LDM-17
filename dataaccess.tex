% section 5.1.1: Data Access

\subsubsection{Data Access}

The data access framework was upgraded for DC3a.

A new {\tt PropertySet} class replaced the old {\tt DataProperty} class.
By simplifying the interface and unifying concepts, it helped to reduce
complexity for developers, remove dependency issues, and bring
{\tt Policy} and {\tt PropertySet} into closer alignment.

The persistence framework saw a few additions.  New formatters for PSFs
and their underlying kernels and spatial functions were written,
allowing those classes to be persisted to Boost archive files and XML
files.  A central, consistent facility for {\tt LogicalLocation} strings
was provided, allowing information from an event, other {\tt Clipboard}
items, or a {\tt Policy} to be substituted into them with optional
formatting.

Database access was improved by implementing the {\tt DbStorage}
portability layer directly in terms of the MySQL API instead of using
CORAL/SEAL as a second level of indirection, by better specification of
database authentication credentials, by the added ability to execute raw
SQL statements (allowing some computations to be done entirely in the
database server), and by the added ability to use expressions and
multi-part field names in queries.

The {\tt DateTime} class that is used as a utility to deal with dates and
times throughout the LSST software stack had its interface significantly
revamped to improve the clarity of the timebase (UTC or TAI) used for
its arguments and results.  (The internal representation of times always
uses TAI.)  Conversions from {\tt DateTime} to and from ISO8601-format
strings were implemented.

