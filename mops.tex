% Section 6.7: MOPS

\subsection{MOPS}

The task of identifying moving objects in LSST images requires that
we, for newly discovered objects, determine their orbits, and for
known moving objects, predict their positions at time an exposure is
made.  This task is handled by the Moving Objects Prediction System
(MOPS).  For operational purposes, we break this into two sides.  {\it
  DayMOPS} is responsible for updating the orbital parameters for all
moving objects based on the latest observations; this is a 
compute-intensive operation that can run during the day.  {\it
  NightMOPS}, given the latest known orbits, will predict the
positions of the objects that intersect a particular image.  Given
that the defining input to {\it NightMOPS} is the position of the
image, it cannot be run ahead of time; thus, it must be run in real
time as part of the alert production.  

The DC2 report provides a summary of MOPS with a particular focus on
NightMOPS.  In DC3a, NightMOPS underwent a number of significant 
upgrades.  The basic algorithm did not change, however; it still involves:
\begin{itemize}
    \item identifying which known MovingObject orbits might be intersecting a 
          given field of view at a given time,
    \item computing precise positions for those MovingObjects at that
      time, and 
    \item Passing those accurate positions, with errors, to the Association 
          pipeline.
\end{itemize}

DC3a introduced two main changes.  First, the identification of
candidate MovingObjects now uses KD-Tree searches implemented by the
Auton FieldProximity code instead of simple quadratic interpolation of
the orbit with a boundary check.  Second, the computation of precise
ephemeris is now done using the JPL Horizons software, the ``gold
standard'' for this type of computations.  Previously, positions were
computed using quadratic interpolation between pre-computed midnight
ephemeris.  These changes have the effect of improving the scientific
quality of the results and the accuracy and precision of the predicted
position and associated errors. 

A newer version of known Solar System orbits, produced by the PanSTARRS MOPS 
team using observations provided by the Minor Planet Center, was used to
\begin{itemize}
    \item Provide the initial catalog of known MovingObjects.
    \item Pre-compute midnight ephemeris used by FieldProximity.
\end{itemize}









