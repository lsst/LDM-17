% Section 6.7: MOPS

\subsection{MOPS}

The version of NightMOPS developed for DC3a presents a number of significant 
upgrades over DC2.

The basic algorithm did not change. It still involves
\begin{itemize}
    \item Identifying which known MovingObject orbits might be intersecting a 
          given field of view at a given time.
    \item Computing precise positions for those MovingObjects at that time.
    \item Passing those accurate positions, with errors, to the Association 
          pipeline.
\end{itemize}

However for DC3a
\begin{itemize}
    \item The identification of candidate MovingObjects uses KD-Tree searches 
          implemented by the Auton FieldProximity code instead of simple 
          quadratic interpolation of the orbit with a boundary check.
    \item The computation of precise ephemeris is done using JPL Horizons 
          software, the golden standard for this type of computations. 
          Previously, positions were computed using quadratic interpolation 
          between pre-computed midnight ephemeris.
\end{itemize}
These changes have the effect of improving the scientific quality of the results
and the accuracy and precision of the predicted position and associated errors.

A newer version of known Solar System orbits, produced by the PanSTARRS MOPS 
team using observations provided by the Minor Planet Center, was used to
\begin{itemize}
    \item Provide the initial catalog of known MovingObjects.
    \item Pre-compute midnight ephemeris used by FieldProximity.
\end{itemize}









