% Section 6.3: Source Detection

\subsection{Source Detection}
\label{secDetection}

\subsubsection{Classes}

The namespace {\tt lsst::afw::detection} (in the {\tt afw} package)
contains the framework classes that enable the detection of
astronomical sources in images.  The {\it meas\_pipeline} package uses
these classes to provide source detection various stages of the IPSD
pipeline, including source detection for WCS determination as well as
final source detection from the difference images.  The fundamental
classes for detection are: 

\begin{description}
    \item[{\tt Threshold}] A threshold level.
    \item[{\tt Footprint}] A set of pixels (internally stored as Spans).
    \item[{\tt DetectionSet}] An STL collection of \code{Footprint}s associated 
        with a \code{MaskedImage}.
    \item[{\tt FootprintFunctor}] Utility for applying a function over the pixels of
        a \code{Footprint}.
\end{description}

An important input to source detection is the model for the
point-spread function (PSF).  This model is provided via the {\tt PSF}
class (in the {\it meas\_algorithms} class; see \Sec{secPsf}). 

\subsubsection{Algorithm Description}

The fundamental class in source detection is a \code{Footprint}, a set of 
connected pixels above (or below) some threshold. Two pixels are said to be 
connected if they touch along a side or by a corner. The internal 
representation of a \code{Footprint} is an STL container of \code{Spans} --- 
a triple of a row number in the image and a starting and ending column. The 
user does need to be aware of this and can use a \code{FootprintFunctor} to
iterate over the pixels in the \code{Footprint}

A \code{Footprint} or set of \code{Footprint}s only makes sense in the 
context of an image; a \code{DetectionSet} associates an STL container of 
\code{Footprint}s with a \code{MaskedImage}, and may be constructed from 
the \code{MaskedImage} and a \code{Threshold}; it is templated over the 
\code{Image} and \code{Mask} pixel types.

A \code{Threshold} may be constructed to be positive or negative value in 
units of flux, variance or standard deviation. The interesting question of what value to 
adopt for this threshold has thus been abstracted from 
the algorithms and is answered separately by \code{Policy} parameters. 

In DC3a we detect on both difference images and on stacked images, which are 
first smoothed using \code{afw::math::Kernel}. In either use case, the output
is a \code{DetectionSet}.

